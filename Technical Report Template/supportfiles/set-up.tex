%preamble document set-up

%-----------------------------------------------------

%Package Includes
\usepackage{lipsum}
%Generates blind text
\usepackage{graphicx}
%Inclusion of graphics, including resizing etc.
\usepackage[top=2.5cm,bottom=2cm]{geometry}
%For page layout, etc.
\usepackage{amssymb}
%Symbols
\usepackage{setspace} 
%Set linespacing
\usepackage[hidelinks]{hyperref} 
%Clickable links, i.e. \ref
\usepackage[toc,nonumberlist]{glossaries} 
%Glossary Package
\makeglossary 
%Needed to make the glossary
\usepackage{todonotes} 
%For making to-do notes in the pdf
\usepackage{float} 
%Redefine the float, now we can use 'H' instead of 'h'
\usepackage{array} 
%Extend the array environment
\usepackage{pdfpages} 
%Allows the inclusion of external pdf's 
\usepackage{color, colortbl}
%Allows us to manage the colour of items in the final output 
\usepackage{fancyhdr}
%Allows control of the headers and footers
\usepackage{nameref}
%Refer to sections by their name
\usepackage{chemfig} 
%Draw chemical diagrams
\usepackage{listings} 
%Typeset programming languages
\lstset{language=bash, numbers=left, frame=single, breaklines=true, numberstyle=\footnotesize}
%Set-up date for listings package
\usepackage{framed} 
%Framed or shaded regions that can break across pages
\usepackage[os=win]{menukeys}
%Format menu sequences, paths and keystrokes from lists
\usepackage{draftwatermark} 
%Put a grey textual watermark on document pages
\usepackage{tikz} 
%Used for drawing within the package
\usetikzlibrary{positioning,decorations.markings,patterns}
\usepackage{multirow} 
%Create tabular cells spanning multiple rows
\usepackage[scientific-notation=true]{siunitx} 
%For putting certain scientific notation i.e. x10^.
\usepackage{titlesec} 
%Custom titles
\usepackage[normalem]{ulem}
%Strikethrough text with \sout{}

%-----------------------------------------------------
%Paragraph and Chapter Styling,

%These are the two commands that allow subsubsections to have numbering and show in the T.O.C.
%The latex numbering system is as follows:

% -1 Part
% 0 Chapter
% 1 Section
% 2 Subsection
% 3 Subsubsection
% 4 Paragraph
% 5 Subparagraph

\titleformat*{\section}{\LARGE\bfseries}
\titleformat*{\subsection}{\Large\bfseries}
\titleformat*{\subsubsection}{\large\bfseries}

%\setcounter{tocdepth}{1} % Show sections
%\setcounter{tocdepth}{2} % + subsections
\setcounter{tocdepth}{3} % + subsubsections
%\setcounter{tocdepth}{4} % + paragraphs
%\setcounter{tocdepth}{5} % + subparagraphs

%-----------------------------------------------------

%Page and Header Styling
\setlength{\headheight}{14pt}
\pagestyle{fancy}
\renewcommand{\headrulewidth}{0.25pt}
\fancyhead[L]{\textbf{NAME} - \textit{ID NO}}
\fancyhead[R]{DEPARTMENT - COMPANY}
\setlength\voffset{0in}

%-----------------------------------------------------

%Watermark
\SetWatermarkText{DRAFT}
\SetWatermarkScale{4}
\SetWatermarkLightness{0.8}
%\SetWatermarkColor{red}

%-----------------------------------------------------

%Document meta-data

\hypersetup{
    pdftitle={TITLE HERE},
    pdfauthor={Author},
    pdfsubject={Subject},
    pdfcreator={Author},
	pdfstartview={XYZ null null 1.00},
}

%-----------------------------------------------------

%reduce pdf size

\pdfminorversion=5
\pdfobjcompresslevel=2
    
%-----------------------------------------------------

%Custom Commands

\definecolor{Grey}{gray}{0.95}
\lstset{language=bash, numbers=left, frame=single, breaklines=true, numberstyle=\footnotesize}

%-----------------------------------------------------

%Include Glossary

%\input{./supportfiles/gloss}

%-----------------------------------------------------

%Tiny Leaves

\newcounter{leafs}
 
\tikzset{%
leaf/.style={/utils/exec=\setcounter{leafs}{0},
    decorate,decoration={
        markings,
        mark=at position 0 with {
            \draw [bend left=35,fill=black] (0,0) to ++(-0.3,0.02*\sign) to ++(0.3,-0.02*\sign);},
        mark=between positions 0.02 and 0.5 step 0.05 with {
            \draw [bend right=35,fill=black] (0.01*\sign,0.01*\sign) to ++(-0.2-\mult,0.3*\sign+\mult*\sign) to ++(0.2+\mult,-0.3*\sign-\mult*\sign);
            \stepcounter{leafs};},
        mark=between positions 0.54 and 0.99 step 0.04875 with {
            \addtocounter{leafs}{-1}
            \draw [bend left=35,fill=black] (-0.01*\sign,0.01*\sign) to ++(0.2+\mult,0.3*\sign+\mult*\sign) to ++(-0.2-\mult,-0.3*\sign-\mult*\sign);},
    }
}}
 
\newcommand{\leaves}[1]{%
\setlength{\lineskip}{6pt plus 6pt minus 0pt}\lineskiplimit=\baselineskip
\def\mult{0.03*\theleafs}
\begin{tikzpicture}
    \node [align=center] (box) {\uppercase{#1}};
    \node [below left=of box] (ll) {};
    \node [above left=of box] (ul) {};
    \node [below right=of box] (lr) {};
    \node [above right=of box] (ur) {};
    \def\sign{-1}
    \draw [bend right=45, fill=black,postaction=leaf] (ul) to (ll) to ++(0.1,0.1) to [bend left=45] (ul);
    \def\sign{1}
    \draw [bend left=45, fill=black,postaction=leaf] (ur) to (lr) to ++ (-0.1,0.1) to [bend right=45] (ur);
\end{tikzpicture}
}
